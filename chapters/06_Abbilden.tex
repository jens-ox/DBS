\section{Abbilden --- ER zu Relational}
\label{sec:abbilden}

\begin{figure}[H]\centering\label{Abbilden}\includegraphics[width=.5\linewidth]{Abbilden}\end{figure}

\paragraph{Abbildungsziel: Kapazitätserhaltende Abbildung}
\begin{itemize}
	\item In beiden Fällen gleich viele Instanzen darstellbar
	\item \textbf{Zu Vermeiden:}
	\item Kapazitätserhöhend: relational mehr darstellbar als mit ER
	\item Kapazitätsvermindernd: relational weniger darstellbar als mit ER
\end{itemize}

\paragraph{Abbildungsregeln}
\begin{itemize}
	\item Entity-/Beziehungstypen \( \leadsto \) Relationenschemata \\* Attribute \( \leadsto \) Attribute Relationenschema \\* Schlüssel \( \leadsto \) übernehmen
	\item Kardinalitäten \( \leadsto \) Schlüsselwahl
	\item Ggf. Relationenschemata und Entity-/Beziehungstypen verschmelzen
	\item Einführung neuer Fremdschlüsselbedingungen: \\*
		- Teil der Schema-Definition \\*
		- Entstehen bei Abbildung von Relationships \\*
		- Ersetzen Linie von Relationship zu Entity
	\item Beziehungstyp \( \leadsto \) Relationenschema mit Attributen des Beziehungstyps und Primärschlüssel der beteiligten Entity-Typen
\end{itemize}
\begin{figure}[H]\centering\label{Abbilden2}\includegraphics[width=.5\linewidth]{Abbilden2}\end{figure}

\begin{fragen}
	\item Warum gibt es im ER-Modell keine Fremdschlüssel?
	\item Was bedeutet ``kapazitätserhaltende Abbildung''? Geben Sie Beispiele.
	\item Wiedergabe der unterschiedlichen Beziehungsabbildungen (1:1, 1:n, m:n)
	\item In welchen Fällen lässt sich das Schema optimieren? Was bedeutet Optimierung hier?
	\item Wie lassen sich mengenwertige Attribute abbilden?
	\item Warum ist Abbildung der folgenden Konstrukte vom ER-Modell ins Relationenmodell problematisch? Rekursive Beziehungen, Partitionierung, Generalis.
\end{fragen}